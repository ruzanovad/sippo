\documentclass[a4paper]{article}
\usepackage{cmap}
\usepackage[utf8]{inputenc}
\usepackage[T2A]{fontenc}
\usepackage{amsfonts}
\usepackage{amsmath, amsthm}
\usepackage{amssymb}
% \usepackage{hyperref}
% \usepackage{multicol}
% \usepackage{xcolor}

\newcommand\letsymbol{\mathord{\sqsupset}}
\usepackage[russian]{babel}
\renewcommand\qedsymbol{$\blacktriangleright$}
\newtheorem{theorem}{Теорема}[section]
\newtheorem{lemma}{Лемма}[section]
\theoremstyle{definition}
\newtheorem*{example}{Пример}
\newtheorem*{definition}{Определение}
\newtheorem*{statement}{Утверждение}
\theoremstyle{remark}
\newtheorem*{remark}{Замечание}

\setlength{\topmargin}{-0.5in}
\setlength{\oddsidemargin}{-0.5in}
\textwidth 185mm
\textheight 250mm

\begin{document}
\section*{Задача о минимизации затрат на пересылку данных по компьютерной сети}

По компьютерной сети из компьютеров $C_1, C_2, \dots, C_n$ требуется за секунду переслать Z МБайт с компьютера $C_1$ на $C_n.$ Известны максимальные пропускные способности $D_{ij}$ МБайт/c для всех имеющихся каналов связи между компьютерами, конфигурация сети, а также стоимость $A_{ij}$ пересылки 1 Мбайта с компьютера $C_i$ на $C_j$.

Составить схему пересылки данных с $C_1$ на $C_n$, при которой пропускная способность ни одного канала не превышена и суммарные затраты на пересылку минимальны.

\[\begin{cases}
		f = \sum\limits_{i = 1}^n\sum\limits_{j=1}^n A_{ij} x_{ij}\to \min                        \\
		\sum\limits_{j = 1}^n x_{1j} = Z                                                   \\
		\sum\limits_{j = 1}^n x_{ij} - \sum\limits_{j = 1}^n x_{ji} = 0, \; i = \overline{2, n-1} \\
		x_{ij} \le D_{ij}, i = \overline{1, n}, j = \overline{1, n}                 \\
		x_{ij}\ge 0, i = \overline{1, n}, j = \overline{1, n}
	\end{cases}\]

$$A = \begin{pmatrix}
		0  & 127 & 83 \\
		46 & 0   & 34 \\
		53 & 129 & 0  \\
	\end{pmatrix}, \; D = \begin{pmatrix}
		0  & 50 & 29 \\
		50 & 0  & 7  \\
		32 & 9  & 0  \\
	\end{pmatrix}, \; Z = 33$$
\paragraph*{Решение}
Имеем следующую задачу ЛП:

\[\begin{cases}
		f = 127 x_{12} + 83 x_{13} + 46 x_{21} + 34 x_{23} + 53 x_{31} + 129 x_{32} \to \min \\
		x_{12}+x_{13} = 33                                                                   \\
		x_{21} + x_{23} - x_{12} - x_{32} = 0                                                \\
		x_{11}\leqslant 0                                                                    \\
		x_{12}\leqslant 50                                                                   \\
		x_{13}\leqslant 29                                                                   \\
		x_{21}\leqslant 50                                                                   \\
		x_{22}\leqslant 0                                                                    \\
		x_{23}\leqslant 7                                                                    \\
		x_{31}\leqslant 32                                                                   \\
		x_{32}\leqslant 9                                                                    \\
		x_{33}\leqslant 0                                                                    \\
		x_{ij} \geqslant 0, \; i = \overline{1, 3}, j = \overline{1, 3}
	\end{cases}\]
Приведем к каноническому виду:
\[\begin{cases}
		f = -127 x_{12} - 83 x_{13} - 46 x_{21} - 34 x_{23} - 53 x_{31} - 129 x_{32} \to \max \\
		x_{12}+x_{13} = 33                                                                    \\
		x_{21} + x_{23} - x_{12} - x_{32} = 0                                                 \\
		x_{11} + s_{1} =  0                                                                   \\
		x_{12} + s_{2} =  50                                                                  \\
		x_{13} + s_{3} =  29                                                                  \\
		x_{21} + s_{4} =  50                                                                  \\
		x_{22} + s_{5} =  0                                                                   \\
		x_{23} + s_{6} =  7                                                                   \\
		x_{31} + s_{7} =  32                                                                  \\
		x_{32} + s_{8} =  9                                                                   \\
		x_{33} + s_{9} =  0                                                                   \\
		x_{ij}, s_k \geqslant 0, \; i = \overline{1, 3}, j = \overline{1, 3}, k = \overline{1, 9}
	\end{cases}\]
Тут достаточно гауссовских преобразований, чтобы получить специальную ЗЛП.

\[\begin{cases}
		f = -3099 + 12x_{23}- 175 x_{22} - 34 x_{23} - 53 x_{31} - 90 s_3\to \max \\
		x_{12}-s_3 = 4                                                            \\
		x_{21} + x_{23} - x_{32} - s_3 = 4                                        \\
		x_{11} + s_{1} =  0                                                       \\
		s_{2} + s_3 =  46                                                         \\
		x_{13} + s_{3} =  29                                                      \\
		-x_{23} + x_{32} + s_{3} + s_4 =  46                                      \\
		x_{22} + s_{5} =  0                                                       \\
		x_{23} + s_{6} =  7                                                       \\
		x_{31} + s_{7} =  32                                                      \\
		x_{32} + s_{8} =  9                                                       \\
		x_{33} + s_{9} =  0                                                       \\
		x_{ij}, s_k \geqslant 0, \; i = \overline{1, 3}, j = \overline{1, 3}, k = \overline{1, 9}
	\end{cases}\]
Базисные переменные - $[x_{12}, x_{21}, x_{11}, s_2, x_{13}, s_4, x_{22}, s_6, s_7, s_8, x_{33}]$

Итерация симплекс-метода:

\begin{table}[h]
	\begin{tabular}{|l|l|l|l|l|l|l|l|l|l|l|l|l|l|l|l|l|l|l|l|l|l|l|l|l|}
		\hline
		$B$      & $x_0$ & $x_{11}$ & $x_{12}$ & $x_{13}$ & $x_{21}$ & $x_{22}$ & $x_{23}$ & $x_{31}$ & $x_{32}$ & $x_{33}$ & $s_1$ & $s_2$ & $s_3$ & $s_4$ & $s_5$ & $s_6$ & $s_7$ & $s_8$ & $s_9$ \\ \hline\hline
		f        & -3099 & 0        & 0        & 0        & 0        & 0        & -12      & 53       & 175      & 0        & 0     & 0     & 90    & 0     & 0     & 0     & 0     & 0     & 0     \\ \hline\hline
		$x_{12}$ & 4     & 0        & 1        & 0        & 0        & 0        & 0        & 0        & 0        & 0        & 0     & 0     & -1    & 0     & 0     & 0     & 0     & 0     & 0     \\ \hline
		$x_{21}$ & 4     & 0        & 0        & 0        & 1        & 0        & 1        & 0        & -1       & 0        & 0     & 0     & -1    & 0     & 0     & 0     & 0     & 0     & 0     \\ \hline
		$x_{11}$ & 0     & 1        & 0        & 0        & 0        & 0        & 0        & 0        & 0        & 0        & 1     & 0     & 0     & 0     & 0     & 0     & 0     & 0     & 0     \\ \hline
		$s_{2}$  & 46    & 0        & 0        & 0        & 0        & 0        & 0        & 0        & 0        & 0        & 0     & 1     & 1     & 0     & 0     & 0     & 0     & 0     & 0     \\ \hline
		$x_{13}$ & 29    & 0        & 0        & 1        & 0        & 0        & 0        & 0        & 0        & 0        & 0     & 0     & 1     & 0     & 0     & 0     & 0     & 0     & 0     \\ \hline
		$s_4$    & 46    & 0        & 0        & 0        & 0        & 0        & -1       & 0        & 1        & 0        & 0     & 0     & 1     & 1     & 0     & 0     & 0     & 0     & 0     \\ \hline
		$x_{22}$ & 0     & 0        & 0        & 0        & 0        & 1        & 0        & 0        & 0        & 0        & 0     & 0     & 0     & 0     & 1     & 0     & 0     & 0     & 0     \\ \hline
		$s_6$    & 7     & 0        & 0        & 0        & 0        & 0        & 1        & 0        & 0        & 0        & 0     & 0     & 0     & 0     & 0     & 1     & 0     & 0     & 0     \\ \hline
		$s_7$    & 32    & 0        & 0        & 0        & 0        & 0        & 0        & 1        & 0        & 0        & 0     & 0     & 0     & 0     & 0     & 0     & 1     & 0     & 0     \\ \hline
		$s_8$    & 9     & 0        & 0        & 0        & 0        & 0        & 0        & 0        & 1        & 0        & 0     & 0     & 0     & 0     & 0     & 0     & 0     & 1     & 0     \\ \hline
		$x_{33}$ & 0     & 0        & 0        & 0        & 0        & 0        & 0        & 0        & 0        & 1        & 0     & 0     & 0     & 0     & 0     & 0     & 0     & 0     & 1     \\ \hline
	\end{tabular}
\end{table}
Таблица не оптимальна (есть отрицательные числа в строке $f$), не является неразрешимой (нет столбцов, состоящих только из отрицательных чисел).

Столбец $x_{23}$ является ведущим, т.к в строке $f$ отрицательное число

Строка $x_{21}$ является ведущей, т.к $\frac{4}{1} = \min_{a_{0q}>0} \frac{a_{i0}}{a_{iq}} = \min \{4, 7\}$

Преобразование...
\begin{table}[h]
	\begin{tabular}{|l|l|l|l|l|l|l|l|l|l|l|l|l|l|l|l|l|l|l|l|l|l|l|l|l|}
		\hline
		$B$      & $x_0$ & $x_{11}$ & $x_{12}$ & $x_{13}$ & $x_{21}$ & $x_{22}$ & $x_{23}$ & $x_{31}$ & $x_{32}$ & $x_{33}$ & $s_1$ & $s_2$ & $s_3$ & $s_4$ & $s_5$ & $s_6$ & $s_7$ & $s_8$ & $s_9$ \\ \hline\hline
		f        & -3051 & 0        & 0        & 0        & 12       & 0        & 0        & 53       & 163      & 0        & 0     & 0     & 78    & 0     & 0     & 0     & 0     & 0     & 0     \\ \hline\hline
		$x_{12}$ & 4     & 0        & 1        & 0        & 0        & 0        & 0        & 0        & 0        & 0        & 0     & 0     & -1    & 0     & 0     & 0     & 0     & 0     & 0     \\ \hline
		$x_{21}$ & 4     & 0        & 0        & 0        & 1        & 0        & 1        & 0        & -1       & 0        & 0     & 0     & -1    & 0     & 0     & 0     & 0     & 0     & 0     \\ \hline
		$x_{11}$ & 0     & 1        & 0        & 0        & 0        & 0        & 0        & 0        & 0        & 0        & 1     & 0     & 0     & 0     & 0     & 0     & 0     & 0     & 0     \\ \hline
		$s_{2}$  & 46    & 0        & 0        & 0        & 0        & 0        & 0        & 0        & 0        & 0        & 0     & 1     & 1     & 0     & 0     & 0     & 0     & 0     & 0     \\ \hline
		$x_{13}$ & 29    & 0        & 0        & 1        & 0        & 0        & 0        & 0        & 0        & 0        & 0     & 0     & 1     & 0     & 0     & 0     & 0     & 0     & 0     \\ \hline
		$s_4$    & 50    & 0        & 0        & 0        & 1        & 0        & 0        & 0        & 0        & 0        & 0     & 0     & 0     & 1     & 0     & 0     & 0     & 0     & 0     \\ \hline
		$x_{22}$ & 0     & 0        & 0        & 0        & 0        & 1        & 0        & 0        & 0        & 0        & 0     & 0     & 0     & 0     & 1     & 0     & 0     & 0     & 0     \\ \hline
		$s_6$    & 3     & 0        & 0        & 0        & -1       & 0        & 0        & 0        & 1        & 0        & 0     & 0     & 1     & 0     & 0     & 1     & 0     & 0     & 0     \\ \hline
		$s_7$    & 32    & 0        & 0        & 0        & 0        & 0        & 0        & 1        & 0        & 0        & 0     & 0     & 0     & 0     & 0     & 0     & 1     & 0     & 0     \\ \hline
		$s_8$    & 9     & 0        & 0        & 0        & 0        & 0        & 0        & 0        & 1        & 0        & 0     & 0     & 0     & 0     & 0     & 0     & 0     & 1     & 0     \\ \hline
		$x_{33}$ & 0     & 0        & 0        & 0        & 0        & 0        & 0        & 0        & 0        & 1        & 0     & 0     & 0     & 0     & 0     & 0     & 0     & 0     & 1     \\ \hline
	\end{tabular}
\end{table}

Проверка на оптимальность: все числа в строке $f$ неотрицательны, тем самым имеем оптимальное базисное решение (0, 4, 29, 0, 0, 4, 0, 0, 0, 0, 46, 0, 50, 0, 3, 32, 9, 0) оптимально.

Вернемся к исходной задаче: так как мы при приведении задачи ЛП к канонической форме
умножали целевую функцию на -1, теперь также умножим ответ на -1
\[X^* = \begin{pmatrix}
		0 & 4 & 29 \\ 0 & 0 & 4 \\ 0 & 0 & 0
	\end{pmatrix},\; f^* = -(-3051) =  3051\]
\end{document}