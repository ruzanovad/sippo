\documentclass{article}
\usepackage{cmap}
\usepackage[utf8]{inputenc}
\usepackage[T2A]{fontenc}
\usepackage{amsfonts}
\usepackage{amsmath, amsthm}
\usepackage{amssymb}
\usepackage{enumitem}
% \usepackage{hyperref}
% \usepackage{multicol}
% \usepackage{xcolor}

% \newcommand\letsymbol{\mathord{\sqsupset}}
\usepackage[russian]{babel}
% \renewcommand\qedsymbol{$\blacktriangleright$}
% \newtheorem{theorem}{Теорема}[section]
% \newtheorem{lemma}{Лемма}[section]
% \theoremstyle{definition}
% \newtheorem*{example}{Пример}
% \newtheorem*{definition}{Определение}
% \newtheorem*{statement}{Утверждение}
% \theoremstyle{remark}
% \newtheorem*{remark}{Замечание}

\setlength{\topmargin}{-0.5in}
\setlength{\oddsidemargin}{-0.5in}
\textwidth 185mm
\textheight 250mm
\renewcommand{\labelitemi}{$\spadesuit$}
\begin{document}

\Large{
\centering
\textsc{\MakeUppercase{Управление в бизнесе}}

% \begin{flushright}
    \raggedleft
    Анри Файоль утверждал, что \underline{пять функции} являются общепринятыми для всех видов управления в бизнесе независимо от целей предприятия:

% \end{flushright}

\begin{itemize}
    \raggedleft
    \item планирование;
    \item организация,
    \item управление,
    \item координация,
    \item контроль.
\end{itemize}
\raggedleft
Со временем первоначальный перечень управленческих функций. сформулированный Файолем, был расширен специалистами по менеджменту до восьми функций:

\begin{enumerate}
    \raggedleft
    \item \textbf{Планирование}. Основная функция менеджмента. Определение целей и направлений деятельности предприятий.
    \item \textbf{Принятие управленческого решения.} Принятие правильного решения в изменяющейся внешней среде является основной проблемой менеджмента.
    \item \textbf{Организация}. Тщательная организация помогает обеспечить эффективное использование человеческих ресурсов. Хорошая организация предполагает структурирование цепи распоряжений, разделения труда, передачу ответственности
    \item \textbf{Укомплектование штата.} Фирмы хороши настолько, насколько хороши в них люди. Менеджмент бизнесом инвестирует рост и развитие преданного, хорошо обученного коллектива.
    \item \textbf{Эффективная коммуникация.} Предприятиям бизнеса необходимо сохранять каналы общения открытыми. Кампании с наилучшим моральным климатом – это те, которые информируют о своих целях и намерениях сотрудников и прислушиваются к ним.
    \item \textbf{Стимулирования.} Вознаграждение сотрудников на основе долговременной программы за нахождение способов эффективной работы.
    \item \textbf{Руководство.} Топ менеджеры становятся признанными лидерами, когда приспосабливают свой стиль руководства к требованиям ситуации.
    \item \textbf{Контроль.} Посредством контрольной функции  антрепренеры сравнивают желаемые результаты с достигнутыми и принимают необходимые корректирующие действия.
\end{enumerate}

}
\end{document}