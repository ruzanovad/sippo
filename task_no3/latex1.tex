\documentclass[a4paper]{article}
\usepackage{cmap}
\usepackage[utf8]{inputenc}
\usepackage[T2A]{fontenc}
\usepackage{amsfonts}
\usepackage{amsmath, amsthm}
\usepackage{amssymb}
% \usepackage{hyperref}
% \usepackage{multicol}
% \usepackage{xcolor}

% \newcommand\letsymbol{\mathord{\sqsupset}}
\usepackage[russian]{babel}
% \renewcommand\qedsymbol{$\blacktriangleright$}
% \newtheorem{theorem}{Теорема}[section]
% \newtheorem{lemma}{Лемма}[section]
% \theoremstyle{definition}
% \newtheorem*{example}{Пример}
% \newtheorem*{definition}{Определение}
% \newtheorem*{statement}{Утверждение}
% \theoremstyle{remark}
% \newtheorem*{remark}{Замечание}

% \setlength{\topmargin}{-0.5in}
% \setlength{\oddsidemargin}{-0.5in}
% \textwidth 185mm
% \textheight 250mm

\begin{document}
Федеральное государственное бюджетное
\\
образовательное учреждение высшего образования
\\
Омский государственный университет им. Ф.М. Достоевского

\vspace*{15\baselineskip}
Задание

ФОРМАТИРОВАНИЕ ТЕКСТА 

В LATEXe \\


(отчет)

\vspace*{11\baselineskip}
Выполнила: Рузанова Д.П.

Проверила: Сергиенко Т.А.
\vspace*{13\baselineskip}

ОМСК 2023
\end{document}